\chapter{Evaluation}
%
\section{GMT characteristics}
%
\subsection{Characteristic values of the oscilloscope}
%
The signal of the GMT can be seen on the oscilloscope screen (s. fig.1-5). fig.1 shows us the zero level \(U_{0}\) and
the amplitude \(\hat{U}\):
%
\begin{align}
    U_{0}   &=  \SI{24}{mV} \pm \SI{40}{mV} \\
    \hat{U} &=  \SI{1,56}{V} \pm \SI{0,04}{V}
\end{align}
%
Further there can be read off the fall time \(t_{f}\) (fig.2), rise time \(t_{r}\) (fig.3), the pulse width \(t_{p}\)
(fig.4) and recovery time \(t_{re}\) (fig.5):
\begin{align}
    t_{f}   &=  \SI{56}{\micro s} \pm \SI{20}{\micro s} \\
    t_{r}   &=  \SI{228}{\micro s} \pm \SI{20}{\micro s} \\
    t_{p}   &=  \SI{108}{\micro s} \pm \SI{20}{\micro s} \\
    t_{re}  &=  \SI{288}{\micro s} \pm \SI{20}{\micro s}
\end{align}
-fig.1-\\
-fig.2-\\
-fig.3-\\
-fig.4-\\
-fig.5-
%
\subsection{Characteristic curve of the GMT}
%
After determining the starting voltage \(U_{start}\) to \(U_{start}=\SI{328}{V}\), the characteristic curve has been
recorded by measuring the count rate over 100 seconds for each step-by-step-increased voltage. The calculated mean value
of the count rate has been plotted versus the mean voltage by SciDAVis:\par
-fig.6-\par
The relative slope in the plateau area is determined to \(m=(3,2 \pm 2,8) \frac{\%}{\SI{100}{V}}\) by SciDAVis.\par
The optimum working voltage \(U_{opt}\) is determined to \(U_{opt}=\SI{500}{V}\), since the maximum possible adjustable
voltage of \(U_{GMT}=\SI{600}{V}\) is still in the plateau area, which starts at \(U_{GMT}=\SI{340}{V}\).
%
\section{Angular dependency of the count rate}
To examine the angular dependency of the count rate the angle of the tube has been varied from \SI[]{-45}[]{\degree} to
\SI[]{+45}[]{\degree} in \SI[]{15}[]{\degree} steps. After measuring the count rate at the optimum working voltage of
\(U_{opt}=\SI{500}{V}\) and calculating the mean value of the 12 count rates of each angle, the count rate was output
versus angle by SciDAVis:\par
-fig.7-\par
%
\section{Absorption characteristics of materials}
%
For investigating the absorption characteristics of materials the count rate has been measured 12 times at
\(U_{opt}=\SI{500}{V}\) for different materials (\(\text{thickness}=\SI{2}{mm}\)) in front of the tube:
%
\begin{align}
    &\text{no absorbing material: }  &&n = \SI{221}{cps} \pm \SI{11}{cps}\\
    &\text{Aluminum (Al): }          &&n = \SI{10}{cps} \pm \SI{4}{cps}\\
    &\text{Lead (Pb): }              &&n = \SI{7}{cps} \pm \SI{2}{cps}\\
    &\text{Tin (Sn): }               &&n = \SI{6}{cps} \pm \SI{2}{cps}\\
    &\text{Acrylic glass: }          &&n = \SI{26}{cps} \pm \SI{5}{cps}\\
    &\text{Cardboard: }              &&n = \SI{104}{cps} \pm \SI{8}{cps}
\end{align}
%
The explanation for the different absorption of the different materials is the mass attenuation coefficient of the
different materials. The mass attenuation coefficient is dependent on the material density \(\rho\) and the attenuation
coefficient \( \mu \). Since the attenuation coefficient depends on the atomic number, the size of the atom does
matter. The reason why there is happening an attenuation is because of the \(\alpha\)-, \(\beta\)- and \(\gamma\)-rays are
interacting with the material atoms. While the \(\alpha\)- and \(\beta\)-rays have a strong interaction with the material,
\(\gamma\)-rays have better penetration due to less likeliness to interact. The smaller the half-value layer of a
material the weaker the penetration. With lead (\isotope[82]{Pb}) and Tin (\isotope[50]{Sn}) and presumably Aluminum
(\isotope[13]{Al}) as well almost all rays are absorbed (dependent on the ray energy). Acrylic glass can probably be
traversed by the \(\gamma\)-rays. Due to there size already a thin sheet of cardboard is sufficient to absorbed
\(\alpha\)-ray particles almost entirely.
%
\section{Counting statistics}
%
For examining the statistics, 90 measurements were taken at a distance of \( \SI{1}{cm} \) and a voltage of
\( U_{opt}=\SI{500}{V} \). The following values for mean \( \bar{n} \), variance \( \sigma^{2} \) and standard deviation
\( \sigma \) were obtained:
%
\begin{align}
    \bar{n}     &=  \SI{4327}{cps} \\
    \sigma^{2}  &=  \SI{4168}{(cps)^{2}} \\
    \sigma      &=  \SI{65}{cps}
\end{align}
The measurement series results a scatter plot in SciDAVis:\par
-fig.8-\par
A histogram made with SciDAVis shows the distribution of the measures:\par
-fig.9-\par
With radioactive decay it cannot be said which atom will decay when. Since it is a statistical process, the histogram
can be described with a Poisson-distribution.
%
\section{Background radiation}
%
To investigate the background radiation the count rate was measured 90 times with no radiation source at a voltage of
\( U_{opt}=\SI{500}{V} \). The measurement stats are:
%
\begin{align}
    \bar{n}     &=  \SI{3}{cps} \\
    \sigma^{2}  &=  \SI{20}{(cps)^{2}} \\
    \sigma      &=  \SI{4}{cps}
\end{align}
The measurement number was plotted versus the count rate by SciDAVis:\par
-fig.10-\par
The associated distribution histogram was drawn by SciDAVis:\par
-fig.11-\par
This histogram can also be described by a Poisson-distribution for the same reason as histogram of the counting
statistics.
%
\section{Natural radioactivity}
%
The radioactivity of a package Brazil nuts should have been determined by measuring the count rate. For that the GMT was
put in a beaker filled with Brazil nuts for 900 seconds at \( U_{opt}=\SI{500}{V} \).\par
The mean value of \( \bar{n}=\SI{3}{cps} \) indicates that the radioactivity of the Brazil nuts is very weak because it
has the same mean value as the background radiation. However, this does not have to mean that the Brazil nuts do not emit
any radiation. Because the tube was very close to the nuts, the background radiation could not pass directly through the
nuts, which implies that the rest of the radiation must have come from the Brazil nuts.\par
-fig.12-