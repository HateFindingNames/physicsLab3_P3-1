\chapter{Execution}
\section{GMT Characteristics}
In this experiment the characteristics of the Geiger-Müller tube are investigated. For this purpose, the protective cap
is first removed from the mica window. The radioactive sample is then removed from its protective enclosure and clamped
in the holder provided on the mounting plate. The distance between the front edge of the radioactive sample and the front edge
of the Geiger-Müller tube is set to \(d = \SI[]{5}[]{cm}\). For this purpose, a steel ruler is used as displayed in the
lab. If not already done, connect the probe to the test point and ground and switch on the oscilloscope. It is
adjusted so that it triggers correctly. The rotary wheel of the potentiometer is turned to center position. The vertical
scaling on the oscilloscope is changed so that an observed peak maxes out the height of the screen to allow a more
accurate reading. The oscillogram is photographed for documentation purposes.\par
The potentiometer is now turned to the counter-clockwise stop. This corresponds to \(U_{GMT} = \SI[]{200}[]{V}\). The LCD
does not yet show any counts at this voltage. The potentiometer is turned up in appropiate increments and the registered
counts are recorded as shown on the LCD. The step size is adjusted accordingly. A small step size is selected for the
range in which the number of pulses increases sharply. A large step size is selected on the plateau. The size of the
steps used can be taken from the tab / graph \ref{}. $U_{Start}$ is the voltage at which the first pulse can be detected % reference missing
on the oscilloscope.
%
\section{Angular Dependency of the Count Rate}
To investigate the influence of the alignment of the GMT, it is set to a distance of \(d = \SI{5}{cm}\) from the
radioactive source. Locking points are already provided on the mounting plate in the range \SI[]{-45}[]{\degree} to
\SI[]{+45}[]{\degree} with increments of \SI[]{15}[]{\degree} each. The radioactive sample remains untouched throughout
the entire experiment. The GMT is inserted into the various locking points one after the other while keeping aligned to
the marks on the mounting plate hence facing the mica window toward the radioactive sample. In each position 12 samples
are taken with a fixed capture time of \SI[]{10}[]{s} each.
%
\section{Absorption Characteristics of Materials}
In this experiment the shielding effect of different samples is investigated. For this purpose the GMT is aligned in
\SI[]{10}[]{\degree} position. There is a slot on the mounting plate into which the test materials can be inserted. The
materials are:
\begin{enumerate}
	\item Aluminium
	\item Lead
	\item Tin
	\item Acylic glass
	\item Cardboard
\end{enumerate}
The thickness of the materials is \SI[]{2}[]{mm} each. For every material 12 measurements are taken. % hab in Zeile 24 fixed eingefügt und denke, damit muss in zukunft die messdauer nicht mehr erwähnt werden.
%
\section{Counting Statistics} \label{sec:count_stat}
For this experiment it is useful to achieve a high count rate. The GMT is placed back in \SI[]{0}[]{\degree} position.
This time the distance is reduced to \(d = \SI[]{1}[]{cm}\). To increase the counting rate of the GMT the voltage is
increased. A voltage of \(U_{GMT} = \SI[]{500}[]{V}\) is used as this voltage should lie within the working area of the
GMT. 90 measurements are taken.
%
\section{Background Radiation}
To measure the background radiation, the experiment described in \cref{sec:count_stat} is repeated. Unlike
before, this time the radioactive source is removed and brought back into the protective vessel.
%
\section{Natural Radioactivity}
%
The radioactivity of Brazil nuts is investigated. For this purpose a beaker is filled with Brazil nuts and the
GMT is carefully inserted into the glass with the mica window facing the nuts. 90 measurements are recorded with the
voltage remaining \(U_{GMT} = \SI[]{500}[]{V}\) % nutty beaker :D.